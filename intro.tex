\section{Introduction}\label{sec:intro}
\epigraph{If you would understand anything, observe its beginning and its development}{Aristotle}

Questions regarding the history of data visualization are (or at least should be) of great importance to historians of science,
to current developers of graphical methods for statistical analysis and the related info-vis community, as well
those just interested in the history of ideas.
In the history of science, diagrams, graphs, maps and other visualizations have often played important roles in discoveries
that arguably might not have been achieved otherwise.%
\footnote{
	Some salient examples are:
	Francis Galton's 1861
	discovery of anti-cyclonic movement of wind around low-pressure areas from contour maps;
	Edward Maunder's ``butterfly diagram'' of the variation of sunspots over time leading to the
	discovery of the ``Maunder minimum,'' from 1645--1715;
	and Henry Moselely's 1913 discovery of the concept of atomic number,
	based largely on graphical analysis (a plot of serial numbers of the elements vs.
	square root of frequencies from their X-ray spectra).
}
At the same time,
in the fields of statistical graphics and information visualization,
developers often create ``new'' methods without any appreciation that they have deep roots in the past.


These two perspectives provided the motivation for the development of the Milestones Project.
This stemmed from the fact that historical accounts of events, ideas and techniques that
relate \emph{inter alia} to modern data visualization were fragmented and
scattered across a wide number of fields.%
\footnote{
Among these are general histories
in the fields of probability \citep{Hald:1990}, statistics
\citep{Pearson:1978,Porter:1986,Stigler:1986}, astronomy \citep{Riddell:1980}, cartography \citep{WallisRobinson:87}.
More specialized accounts focus on the early history of graphic recording
\citep{HoffGeddes:1959,HoffGeddes:1962}, statistical graphs
\citep{Funkhouser:1936,Funkhouser:1937,Royston:1970,Tilling:1975}, fitting equations to empirical data
\citep{Farebrother:1999}, cartography \citep{Friis:1974,Kruskal:1977} and thematic mapping
\citep{FriendlyPalsky:2007,Palsky:1996,Robinson:1982}, and so forth.
}
When this work began in the mid 1990s, there were no accounts or resources that spanned the entire development
of visual thinking and the visual representation of data across different disciplines and perspectives.
The Milestones Project began simply as an attempt to collate these diverse contributions into a single,
comprehensive listing, organized chronologically, and containing representative images, references to
original sources and links to further discussion--- a source for ``One-Stop Shopping'' on the history of
data visualization.

In \secref{sec:project}, we describe the evolution of the Milestones Project. \secref{sec:vistime}
presents some historical and modern approaches to one self-referential question: how can data
visualization be applied to its own history? \secref{sec:historiography} introduces another self-referential
topic we
call \emph{statistical historiography},
which entails the use of statistical and graphical methods for the analysis and understanding of historical innovations, developments, and trends.
But first we give some brief vignettes of historical topics and questions for which the Milestones Project has
proved invaluable in our own research.

\begin{comment}
\begin{itemize*}
 \item The first statistical graph?
 \item Who invented the scatterplot?
 \item The origin of mosaic displays
 \item The Golden Age of statistical graphics
\end{itemize*}
\end{comment}

\subsection{The first statistical graph}
In the history of statistical graphics \citep{Friendly:06:hbook},
as in other artful sciences, there are a number of inventions and developments
that can be considered ``firsts'' in these fields.
The catalog of the Milestones Project
\citep{FriendlyDenis:01} lists 70 events that can be considered to be the
initial use or statement of an idea, method or technique that is now
commonplace, but there is probably no question more fundamental than
that of the first visual representation of statistical data.

\begin{figure}[htb]
 \centering
 \includegraphics[width=\textwidth]{fig/langren-google-overlay}
 \caption{van Langren's 1644 graph, re-scaled and overlaid on a modern map of Europe.
 Toledo is located at
lat/long
%39$^o$51$^'$36$^\"$N, 4$^o$01$^'$48$^\"$W
(\degree{+39.86}N, \degree{-4.03}W), Rome is located at (\degree{+41.89}N, \degree{+12.5}W),
both shown by markers on the map.  This image makes clear what van Langren wished to communicate:
the wide variability of the estimates, but also shows how far the estimates were biased.}%
  \label{fig:langren-google-overlay}
\end{figure}

In \citet{Friendly-etal:2010:langren}
we argue that the 1-dimensional line graph shown in \figref{fig:langren-google-overlay}
by Michael Florent
van Langen \citep{Langren:1644} should be accorded this honor.
The graph shows 12 estimates of the distance in longitude between Toledo and Rome, overlaid on a modern map.
van Langren used this to demonstrate that these estimates were all subject to large errors and to
propose to King Phillip of Spain that only he had a sufficiently precise method for the determination
of longitude for navigation at sea.

The telling of van Langren's story turned out to involve astronomy, archival research,
patronage in the \Cent{17} and even an unsolved problem of cryptography,
but also serves as one example of statistical historiography.  For the present purposes
we note simply that the Milestones Project provided the infrastructure for this research---
a time-based, cross-referenced catalog of images, references and links to related work.

\subsection{Who invented the scatterplot?}
Although there are earlier precursors, the main graphical methods used today---
pie charts, line graphs and bar charts--- are generally attributed to
William Playfair in works around the beginning of the \Cent{19}
\citep{Playfair:1786,Playfair:1801}. All of these are essentially univariate
displays of some aspect of a single variable. The next major invention,
and the first true bivariate display is \scat\, whose use by
\citet{Galton:1886} led to the discovery of
correlation and regression, and ultimately to much of present multivariate
statistics. So, it is perhaps surprising that there is no one widely
credited with the invention of this idea.

In \citet{FriendlyDenis:05:scat} we trace the early origins of ideas related to
the scatterplot, why, in Playfair's time, it was nearly impossible to think about
and visualize bivariate relations, and how Galton's visual insight from
a scatterplot contributed to the rise of modern statistics and graphics.
But, the resources available in the Milestones Project allowed us to attribute
the essential ideas of the scatterplot to J. F. W. Herschel in two 1832 papers.

\subsection{The Golden Age of statistical graphics}

In our initial web presentation of the Milestones Project, it proved convenient
to sub-divide this history of data visualization into epochs, each of which turned
out to be describable by coherent themes.  For reasons we describe later, one
period turned out to be particularly noteworthy, both for the sheer number of
contributions and for the beauty and elegance of their execution.
We call this period, from roughly 1850 to 1900 ($\pm 10$) the Golden Age of
Statistical graphics \citep{Friendly:2008:golden}.

\figref{fig:mileyears4} shows the time distribution of 260 miletone events
listed in the Milestones Project in 2007 together with the labels we used for
epochs.
In \citet{Friendly:2008:golden} we trace the origin of this period in terms
of the infrastructure required to produce this explosive growth of contributions
to data visualization: systematic data collection by state agencies,
the rise of statistical and visual thinking, and enabling  developments of
technology.
%% one figure
\begin{figure}[!htb]
  \centering
  \includegraphics[width=.9\textwidth,clip]{fig/mileyears4}
  \caption{The time distribution of events considered milestones in the history of
  data visualization, shown by a rug plot and
  density estimate.
  The density estimate is based on $n=260$ significant events in the history of data
  of data visualization from 1500--present.
  The developments in the highlighted period, from roughly 1840--1910 comprise the
  Golden Age of statistical graphics.
  }
  \label{fig:mileyears4}
\end{figure}


