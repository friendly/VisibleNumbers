\section{The Milestones Project}\label{sec:project}
\epigraph{Direction is more important than speed. We are so busy looking at our speedometers \\ that we forget the milestone.}{Anonymous} % I like the white space break in the quote.  Also, would "Unknown" be more fitting for the author?

An early overview of the content and aims of the Milestones Project appeared in \cite{Friendly:04:gfkl}. Here we update that description and provide a few technical details on some problems that were encountered in attempting to make the history of data visualization convenient for collecting, browsing, searching, and analysis.

\subsection{Origin, structure and evolution}\label{sec:structure}
The initial step in portraying the history of data visualization was to create a simple chronological listing of milestone items with capsule descriptions, bibliographic references, markers for date, person, place, and links to media assets, related sources and more detailed commentaries. The initial database contained the 105 developments listed by \citet{BenigerRobyn:1978}, and incorporated additional records from \citet{Hankins:1999}, \citet{Tufte:1983,Tufte:1990,Tufte:1997}, \citet{Heiser:2000}, among others.

This began as a single \LaTeX\ file (with markup tags for all relevant bits of information), used to produce a hyper-linked PDF document.  A variety of software tools (perl scripts, Unix utilities) allowed us to turn this single source \emph{directly} into the web version originally shown at
\url{http://www.math.yorku.ca/SCS/Gallery/milestone}.  Other custom software tools allowed us to add new milestones items from text files using a template of tags (DATE:, AUTHOR:, WHAT:, REF:, IMG:, etc.) and extract the information about milestones items, authors, images, etc. in a variety of forms (CSV, XML, JSON) that could be used as input for analyses and graphic displays.  For example, \figref{fig:mileyears4} was produced in SAS software by piping the output of a latex to csv translator:
\begin{verbatim} 
itemdb -o milestones.csv < milestones.tex | sas -i milestones.csv mileyears.sas 
\end{verbatim}

It soon became apparent that such a text-based representation was inadequate. Updating the milestones data required that the \LaTeX\ files be shared among several collaborators, revisions needed to manually verified for consistency and manually versioned; milestone assets, such as images, web links and references were not centrally stored and were therefore not easily accessible by others and often duplicated. Furthermore, updates to the web site required an inefficient number of steps of verification, re-building, and synchronization with the server, all the while being done manually and at different intervals, meaning the website was always out of date.

Around 2005, we began the process of refactoring the process into a more dynamic one; to convert the flat file into a relational database, create a Milestones administration content management system, and completely redesign the Milestones web site public facing user interface. Specifically, we wanted to facilitate contributions by any number of trusted collaborators, and allow for the dissemination of milestones data via an easy-to-browse public user interface.

%\TODO{MF: The language here needs a bit of smoothing, going from a database-centric towards the target audience.}
Migrating the data to this format provided some challenges. First, the organization of the milesontes data needed to be restructured. The data was fragmented across many files and redundancy needed to be minimized. Second, the milestone's data and asset types needed to be to redefined in a way that would facilitate relations between any property, that would be scalable moving forward and could be exported to a any format for analysis.
\begin{comment}
Even I am not quite sure what this first sentence means!
\end{comment}
\begin{comment}
Does the above make more sense?
\end{comment}
In order to do this, we needed to partition the data into its relevant entities: namely the milestone itself, its descriptors, such as its aspect, author, subject, keywords, reference, and linked media assets, such as images. The aspect, author, subject, keyword, and reference descriptors would exist as a many-to-many relationship between it and the milestonei, so that any combination of these items could relate to multiple Milestones, while media assets, on the other hand, can only belong to one milestone at a time. \figref{fig:datavis-schema-2} illustrates these relationships.

\begin{figure}[!htb]
  \centering
  \includegraphics[width=\textwidth,clip]{fig/datavis-schema-3}
  \caption{Simplified schema for the MySQL database for the Milestones Project. The main table (\texttt{milestone}) contains information regarding each of the items considered a milestone in the history of data visualization, linked to other tables (e.g., \texttt{reference}, \texttt{mediaitem}) by unique (primary) keys. Other supporting tables (e.g., \texttt{milestone2aspect}) provide for convenient lookups of descriptors of these milestones items (\texttt{subject}, \texttt{aspect}, \texttt{keyword}).
  }
  \label{fig:datavis-schema-2}
\end{figure}

Normalizing the data in this way enabled us to free the database of modification anomalies; ensured that the database structure was scalable, could be extended with a minimum of modifications, and data exported to a variety of formats. Most importantly, it provides a query-neutral database model \citep{Codd:1971} that can be used to power web presentation, and customized indexed searching. The last major benefit, which will be demonstrated in \secref{sec:vistime}, is that this schema allows for any type of analysis of the Milestones data itself.

At present, the Milestones Project documents 288 contributions, with nearly 350 references, information on 336 authors, and 774 media items, made up of 371 images appearing online on the \url{http://datavis.ca/milestone} site, and 403 hyperlinks to images and documents that are externally hosted. In addition, we maintain an offline image database comprising over 1,100 images collected from various sources. Over time, these too will be incorporated into the database.

\subsection{User interface}
The second challenge related to how to display such a large amount of information to end users, via an easy-to-use user interface that would provide and overview, visual search results, and details about these events in the history of data visualization. To help with the visual presentation, we decided to break down the display into time-based groupings of the milestones content by epochs (Pre-1600, 1600s, 1700s, etc.), each with its own category (e.g., 1600--1699: Measurement and Theory) and descriptive text. The visual design of the interface adopts Ben Shneiderman's mantra: ``Overview first, zoom and filter, then details on demand'' \citep{Shneiderman:1996:IEEE}, ultimately the most effective way to present data with rich metadata. To do this, we used a timeline view (\figref{fig:datavis-timeline2}) of the milestones items displayed on the overview landing page. 

\begin{figure}[!htb]
  \centering
  \includegraphics[width=\textwidth,clip]{fig/datavis-timeline2}
  \caption{Timeline view of the Milestones Project on the site \texttt{http://datavis.ca/milestone}. In this view, the top panel shows a detailed view of the segment of history highlighted in the bottom panel, both of which can be separately scrolled. Items in the top panel show a brief text tag, colour-coded by category. Clicking on an item in this panel brings up a small description, which is further linked to the details of the milestone item.}
  \label{fig:datavis-timeline2}
\end{figure}

This timeline, based on the SIMILE Timeline Widget (\url{http://www.simile-widgets.org/timeline}) allows multiple connected bands of time, showing events at different resolutions. Each band can be separately panned by dragging left or right with the mouse pointer, scroll wheel, or keyboard arrow keys. The timeline view, although most obvious, is just one of several possibilities for a visual overview or interaction with the underlying milestones database data. One Milestone detail pages, the data is dynamically displayed from the same data source via a list view (with drop down quick links). The way the data is organized makes it trivial to add new visual representations. In \secref{sec:geography} we will illustrate how it can be explored using a map-based display.
 
\begin{comment}
D: Below is my stab at the above few paragraphs:

\end{comment}
